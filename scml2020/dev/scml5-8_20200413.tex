\documentclass[12pt]{jarticle}
\usepackage[dvipdfmx]{graphicx}
\usepackage{url}
\usepackage{listings,jlisting}
\usepackage{ascmac}
\usepackage{amsmath,amssymb}

%ここからソースコードの表示に関する設定
\lstset{
  basicstyle={\ttfamily},
  identifierstyle={\small},
  commentstyle={\smallitshape},
  keywordstyle={\small\bfseries},
  ndkeywordstyle={\small},
  stringstyle={\small\ttfamily},
  frame={tb},
  breaklines=true,
  columns=[l]{fullflexible},
  numbers=left,
  xrightmargin=0zw,
  xleftmargin=3zw,
  numberstyle={\scriptsize},
  stepnumber=1,
  numbersep=1zw,
  lineskip=-0.5ex
}
%ここまでソースコードの表示に関する設定

\title{scml5-8}
\author{グループ8\\
  29114003 青山周平\\
}
\date{2020年4月13日}

\begin{document}
\maketitle

\section{5 Financial Reports}
エージェントの財務状況の概要は定期的に公開されます(レポート期間ごとのステップ).
これらのレポートにはエージェントの現金、在庫、過去の違反行為に関する以下の情報が含まれています。

\begin{description}
  \item[Balance] エージェントの残高。(マイナスの残高は破産したエージェントを示します。)
  \item[Inventory] カタログ価格で評価されるエージェントの在庫の値。
  \item[Breach Probability] シミュレーションでこれまでに違反したエージェントの契約の割合。
  \item[Breach Level] エージェントの平均違反レベル。
\end{description}

\section{6 Simulation Steps}
すべてのSCMエージェントは、初期化関数とステップ関数を実装します。
前者は1日目が開始する前に、エージェントの動作の初期化を,シミュレータによって呼び出されます。
初期化後、シミュレータは,次のようなすなわちエージェントのステップ関数を呼び出すループ(図3)を繰り返します。

1.negotiation-speed-multiplier roundsで,すべての登録済みの交渉を実行します。
締結された交渉はその後、signing-delay daysの後に承認される。
さらに、納期付きのexogenous contractsは,来たるべきexogenous contracts horizon daysに,承認のために登録されます。

2.今日(このステップで?),承認するために登録されたすべての契約を,承認するかどうか決める(exogenous onesを含む)。

3.売り手から買い手に製品を提供することや,買い手から売り手に送金することで,今日実行される予定のすべての契約を実行します。
実行できない契約については、潜在的な違反を処理し、破産を報告します。

4.不特定な命令の中で,すべてのエージェントのステップ関数を呼び出します。
そうすることで、新しい交渉要求の引き金となるはずで,それはまた,翌日新しい交渉につながる可能性となる。

5.すべての商品の取引価格とエージェントごとのスポット市場価格の価格ペナルティを更新します。

6.すべての工場のすべてのラインで実行されているすべての製造プロセスの1日の生産をシミュレートします。
完成したすべての製品を在庫に移動します。

7.財務レポートを発行します(trading period daysごとに)。


いくつかの例があります:

\begin{itemize}
\item ありふれた出来事を覚えておく(思い出す?)ために、早朝に煩わしい交渉担当者(exogenous onesを含む)と会議をし,なんとか合意を得るために一生懸命働くことを考えてください。
その後、CEOは(ヨットで)朝食をとりながら,契約になるような,彼らが好む合意だけを承認しにいきます.
トラックは正午までに(渋滞を回避するため)製品を配達し、お金を転送する(つまり、契約を実行する)。
その後、CEOは(ゴルフ中に)翌日の交渉について,交渉者にメッセージを送る.
その後、工場は夜遅くまで稼働します。
毎週金曜日に工場が夜間休業した後,財務報告書が公開される。

\item 製品は契約実行後に製造されるため、エージェントは彼らが製造された同じ日には,製品を販売できません。
でも、購入された日に,素材製品を使用することはできます。
\end{itemize}


\section{7 The SCML Platform}
SCML2019と同様に、SCML2020は、開発用のPythonフレームワークであるNegMAS [7]の上で実行されます。
それは,シミュレーション環境に埋め込まれた自律交渉エージェントです。
SCML2020シミュレーションは徹底的に開発されました.
これは,SCML2020に参加する前に,SCML2019向けに開発されたエージェントを適応させる必要があるということを意味します。
今後は、SCMLの将来的に,SCML2020用に開発されたエージェントとインターフェース互換が何度も使われていくだろう。

\subsection{7.1 Negotiators}
negotiatorは、エージェントに代わって交渉を行うエンティティです。
すべてのnegotiatorは次のインターフェースに従って実装する必要があります:

\begin{description}
  \item[Propose]すべての交渉問題に,価値の割り当てを提案します。
  \item[Respond]オファーに応じて、交渉を受け入れるか、拒否するか、または終了します。
\end{description} 

negotiatorは、代理で契約を交渉する目的でエージェントによって作成される動的エンティティです。
2種類のnegotiatorがサポートされています。

1.権限を与えられた交渉者(Empowered negotiators)は、全面的な意思決定者です。
彼らは作られたときに,ユーティリティ関数を割り当てられます.
彼らはユーティリティ関数を使用してオファーを出し、他の人のオファーに応答します。

2.Pass-through negotiatorsは、オファーとカウンターオファーを、自身を作成したエージェントに渡すだけです。
したがって、意思決定は、controllerと呼ばれる作成エージェントの責任です。
controllerは通常、複数のnegotiatorを管理し、すべてのnegotiatorに対して提案と応答をどのようにするか決定します。
controllerとPass-through negotiatorsは集中型の交渉戦略を実装するためによく利用されています.

\subsection{7.2 Agents (Factory Managers)}
エージェント(factory managerとも呼ばれる)は、SCMの世界でファクトリを制御します。

\subsubsection{7.2.1 Callbacks}
エージェントはさまざまなコールバックを実装できます。
シミュレータは、シミュレーション中の適切なタイミングでそれらを呼び出します.
Onで始まるコールバックの返り値は必要はありません。
それらは単なる参考情報です。
その他のコールバックでは、エージェントが何らかのアクションを実行する必要があります(たとえば、交渉要求に応答するなど)。
最初の2つのコールバックは、シミュレータのメインループによって呼び出されます。

\begin{description}
  \item[On Init] ワールドが初期化された後、シミュレーションが始まる前に呼び出されます。
  \item[On Step] シミュレーションループで呼び出されるステップ。エージェントの交渉ロジックの1つのステップをシミュレートします。
\end{description} 

次のコールバックのセットはイベントによって引き起こされます。名前が示すイベントによってトリガーされます。

\begin{description}
  \item[Respond to Negotiation Request] 別のエージェントがこのエージェントとの交渉を要求するたびに呼び出される交渉要求に応答します。このエージェントは、交渉を同意するかどうかを決めれます。エージェントが同意すると、新しい交渉が登録されます。
  \item[Sign All Contracts] シミュレータによって呼び出されたすべての合意が得られた契約の締結をエージェントに許可します。それによって,これらの契約は拘束力を持ちます。さらに、このコールバックで、潜在的なexogenous agreementsはエージェントに対して明らかにされます.このような交渉された合意と同様に、エージェントは自由に承認できます。
  \item[On Negotiation Success/Failure] エージェントが関与する交渉が終了したときに呼び出されます。
  \item[On Contracts Finalized] 承認済みおよびキャンセル済みの契約の組とともにこのコールバックを呼び出します.契約が締結すれば,それは両方の当事者が承認したということです。いずれかの当事者が署名しない場合はキャンセルされたということです。
  \item[On Agent Bankrupt] エージェントが破産した場合エージェントが破産したと宣言されたときに呼び出され、他のすべてのエージェントに通知します。その破産したエージェントと,将来的に契約の関係にあったエージェントは,無効化された契約(つまり、キャンセルされて実行されない)のリストを受け取ったり,部分的に実行されたり(つまり、当初合意された数量よりも少ない量は,配達時に取引され部分的に実行される)します.
  \item[On Failure] 必要な素材または資金がないために生産が失敗したときに呼び出されます.
\end{description} 

\subsubsection{7.2.2 Actions}
エージェントは次のアクションを使用して交渉と財務を制御します。

\begin{description}
  \item[Request Negotiation] 交渉要求を別のエージェントに送信します。リクエストするエージェントは、この交渉に使用するnegotiatorを指定しなければならない。
  \item[Request Negotiations] 交渉要求のリストをエージェントのリストに送信します。リクエストするエージェントはこのネゴシエーションに使用するnegotiatorsのリスト、またはcontrollerを指定します。
  \item[Schedule Production] 製造プロセスを指定されたラインに追加して、指定された日に開始します。もしラインが指定されていない場合、工場は利用可能なラインを使用します。日が範囲として指定されている場合、工場はその範囲内の日に実行します。スケジュール方法に従って、最新の生産を可能な限り最も早い日にスケジュールします.
  \item[Cancel Production] 生産予定を中止します。
  \item[Set Production] すべてのラインで、特定のステップ(通常は現在)の生産を設定します。
\end{description} 

エージェントは、Agent-World-Interfaceにアクセスして、工場や他のエージェントに関する情報を収集することもできます。利用可能な方法は次のとおりです。

\begin{description}
  \item[Get State] 工場の状態を読み取ります。
  \item[Available For Production] 生産に使用可能な生産スロット(日/ライン)を返します。
  \item[Bulletin Board] 掲示板にアクセスして、製品情報、財務報告、違反リスト、またはシミュレーション設定(日数、当日など)を読み取ります。
\end{description} 

\section{8 Tournament Mechanics}
\begin{description}
  \item[参加方法] サプライチェーンマネジメントリーグ(SCML)に参加するには、factory managerとして機能する自律エージェントのコードを記述して送信することが必要です。生産グラフはSCML2020のチェーンである限り、エージェントが,同一のラインを持つ1つの工場を管理するのに,エージェントは,そのような工場を製造プロファイルで管理するのに十分堅牢でなければなりません(すなわち、工場の割り当てと生産コスト)。それは,特定のプロファイルはシミュレーションごとに異なるため.
  \item[競争方法] SCML2020には2つの別々のトラックがあります。すべてのエージェントは両方のトラックで実行されます。標準トラックでは、各チームのエージェントの最大で1つのインスタンス化が各シミュレーションで実行されます。他の参加者によって準備されたエージェントと組織委員会によって準備されたエージェントの未知の組み合わせ。共謀トラックでは、同じチームのエージェントの複数のインスタンス化が単一のシミュレーション中に実行されます。それぞれのインスタンス化の正確な数はシミュレーションによって異なり、事前に通知されることはありません。このトラックでは、同じエージェントのインスタンスが互いに共謀して,市場の一角を独占したり,他の共謀的な行動を示す追い詰めようとすることは完全に合法です.
  \item[勝つ方法] エージェントのパフォーマンスはスコアによって測定されます。エージェントのスコアは,すべてのシミュレーションを通して増加した利益の中央値です.
  1つのシミュレーション中に工場によって発生した利益は、次のように計算されます。

  $Profit = \frac{B_N + \frac{1}{2}I_N - B_0}{B_0}$

  ここで、$B_0$と$B_N$は、それぞれシミュレーションの開始時と終了時のエージェントのバランスです。$I_N$は、ゲーム終了時のエージェントの在庫にある製品の価値です。この値は、取引価格(式1を参照)ですが、誘発的な取引のために、在庫は取引価格の半分だけで評価されます。それ方法では、製品を買いだめするよりも販売するほうが平均的に収益性が高くなります。
  2つのトラックは、予選ラウンドと最終ラウンドの2つのラウンドで実施されます。そのすべての参加者SCMLのいずれかルールを破ると失格となり、それをしなければ予選に参加できます。その後、予選ラウンドのトップスコアエージェントが最終ラウンドに参加します。最終結果はIJCAI 2020で発表されます。最終候補者は,IJCAI 2020のANACワークショップに代表者を送ることが期待されています.それに参加し、エージェントの説明について簡単なプレゼンテーションを行う機会が与えられます。IJCAI 2020では2つのアワードが発表されます(関連する金銭的報酬とともに).これは,2つのトラック(標準と共謀)に対応します。
\end{description} 

\end{document}
